\documentclass[a4paper, 11pt]{article}

%Comandos para configurar el idioma
\usepackage[spanish,activeacute]{babel}
\usepackage[utf8]{inputenc}
\usepackage[T1]{fontenc} %Necesario para el uso de las comillas latinas.
\usepackage{geometry} % Used to adjust the document margins

%Importante que esta sea la última órden del preámbulo
\usepackage{hyperref}
   \hypersetup{
     pdftitle={Cuestionario de teoría 2},
     pdfauthor={Antonio Álvarez Caballero},
     unicode,
     breaklinks=true,  % so long urls are correctly broken across lines
     colorlinks=true,
     urlcolor=blue,
     linkcolor=darkorange,
     citecolor=darkgreen,
     }

   % Slightly bigger margins than the latex defaults

   \geometry{verbose,tmargin=1in,bmargin=1in,lmargin=1in,rmargin=1in}
\newcommand\fnurl[2]{%
  \href{#2}{#1}\footnote{\url{#2}}%
}

%Paquetes matemáticos
\usepackage{amsmath,amsfonts,amsthm}
\usepackage[all]{xy} %Para diagramas
\usepackage{enumerate} %Personalización de enumeraciones
\usepackage{tikz} %Dibujos
\usepackage{ wasysym } % Para la sonrisa

%Tipografía escalable
\usepackage{lmodern}
%Legibilidad
\usepackage{microtype}

%Código
\usepackage{listings}
\usepackage{color}

\definecolor{dkgreen}{rgb}{0,0.6,0}
\definecolor{gray}{rgb}{0.5,0.5,0.5}
\definecolor{mauve}{rgb}{0.58,0,0.82}
\definecolor{orange}{cmyk}{0,0.4,0.8,0.2}
\definecolor{darkorange}{rgb}{.71,0.21,0.01}
\definecolor{darkgreen}{rgb}{.12,.54,.11}
\definecolor{myteal}{rgb}{.26, .44, .56}
\definecolor{gray}{gray}{0.45}
\definecolor{lightgray}{gray}{.95}
\definecolor{mediumgray}{gray}{.8}

\setlength{\parskip}{.5em} % por defecto el espacio entre párrafos es 0pt




\lstset{frame=tb,
  language=Python,
  aboveskip=3mm,
  belowskip=3mm,
  showstringspaces=false,
  columns=flexible,
  basicstyle={\small\ttfamily},
  numbers=left,
  numberstyle=\tiny\color{gray},
  keywordstyle=\color{blue},
  commentstyle=\color{dkgreen},
  stringstyle=\color{mauve},
  breaklines=true,
  breakatwhitespace=true,
  tabsize=3
}

\title{Cuestionario de teoría 2}
\author{Antonio Álvarez Caballero\\
    \href{mailto:analca3@correo.ugr.es}{analca3@correo.ugr.es}}
\date{\today}

\theoremstyle{definition}
\newtheorem{ejercicio}{Ejercicio}
\newtheorem{cuestion}{Cuestión}
\newtheorem*{solucion}{Solución}
\newtheorem*{bonus}{BONUS}


%%%%%%%% New sqrt
\usepackage{letltxmacro}
\makeatletter
\let\oldr@@t\r@@t
\def\r@@t#1#2{%
\setbox0=\hbox{$\oldr@@t#1{#2\,}$}\dimen0=\ht0
\advance\dimen0-0.2\ht0
\setbox2=\hbox{\vrule height\ht0 depth -\dimen0}%
{\box0\lower0.4pt\box2}}
\LetLtxMacro{\oldsqrt}{\sqrt}
\renewcommand*{\sqrt}[2][\ ]{\oldsqrt[#1]{#2} }
\makeatother

%%%%%%%%%%%%%%%%%%%%%%%%%%%%%%%%%%%%%%%%%%%%%

\begin{document}

  \maketitle

  \section{Cuestiones}

  \begin{cuestion}
    Identificar la/s diferencia/s esencial/es entre el plano afín y el plano
    proyectivo. ¿Cuáles son sus consecuencias? Justificar la contestación.

  \end{cuestion}

  \begin{solucion}
    La principal diferencia es que en el plano afín existen rectas paralelas, mientras
    que en el plano proyectivo todas las rectas se cortan. La principal consecuencia
    es que en el plano afín el paralelismo no se pierde tras una trasnformación
    y en el plano proyectivo sí. Esto nos permite realizar unas operaciones que no serían
    posibles si nos restringimos a transformaciones afines, como tomar fotos o construir
    panoramas.
  \end{solucion}

  \begin{cuestion}
    Verificar que en coordenadas homogéneas el vector de la recta definida por
    dos puntos puede calcularse como el producto vectorial de los vectores de los
    puntos $(l = x \times x’)$. De igual modo el punto intersección de dos rectas $l$ y
    $l’$ está dado por $x = l \times l’$

  \end{cuestion}

  \begin{solucion}
      Sean $x=(a,b)$ e $y=(c,d)$ dos puntos del plano afín. En el plano proyectivo,
      con coordenadas homogéneas, equivalen a, permitiendo la licencia de los nombres,
      $x=(a,b,1)$ e $y=(c,d,1)$. Veamos su producto vectorial.

      \[
      \left(
      \begin{array}{ccc}
          i & j & k \\
          a & b & 1 \\
          c & d & 1 \\
      \end{array}
      \right) = i(b-d) + j(c-a) + k(ad-bc) = (b-d,c-a,ad-bc) = (\frac{b-d}{ad-bc},\frac{c-a}{ad-bc},1) = l
      \] \\

      Ahora comprobemos si dicha recta (dada por ese vector) contiene dichos puntos, es decir,

      \[
        l^T x = 0 \Leftrightarrow \left(\begin{array}{ccc}
                                                  \frac{b-d}{ad-bc} \\
                                                  \frac{c-a}{ad-bc} \\
                                                  1 \\
                                                \end{array}\right) \cdot (a,b,1) = 0
      \]

      \[
        \frac{ab -bd}{ad-bc} + \frac{bc-ab}{ad-bc} + 1 = 0 \Leftrightarrow \frac{bc-ad}{ad-bc} + 1 = 0 \Leftrightarrow -1 + 1 = 0
      \]

      Análogamente se prueba para el punto $y$, verificando también la igualdad. Así que es la recta que buscamos.

      $$\smiley$$


  \end{solucion}

  \begin{cuestion}
    Sean $x$ y $l$ un punto y una recta respectivamente en un plano proyectivo $P1$
    y suponemos que la recta $l$ pasa por el punto $x$, es decir $l^Tx=0$. Sean $x'$ y $l'$
    un punto y una recta del plano proyectivo $P'$ donde al igual que antes $(l')^T x'=0$.
    Supongamos que existe un homografía de puntos $H$ entre ambos planos proyectivos,
    es decir $x'=Hx$. Deducir de las ecuaciones anteriores la expresión para la
    homografía $G$ que relaciona los vectores de las rectas, es decir $G$ tal que $l'=Gl$.
    Justificar la respuesta

  \end{cuestion}

  \begin{solucion}
      Partimos de
      $$(l')^T x' = 0$$
      Usando la homografía $H$ deducimos que
      $$(l')^T Hx = 0$$
      Ahora reagrupamos y lo escribimos como
      $$\big((l')^T H\big)x = 0$$
      Como sabemos que $l^T x = 0$  deducimos
      $$\big((l')^T H\big) = l^T$$
      Trasponemos ambos lados de la ecuación (la traspuesta invierte el orden del producto)
      $$H^T l' = l$$
      Ahora multiplicamos por la inversa de $H^T$ por la izquierda en ambos miembros
      (podemos hacerlo, ya que al ser una matriz homografía es invertible, ver cuestión
      6)
      $$l' = (H^T)^{-1} l$$
      Por tanto nuestra conclusión es
      $$G = (H^T)^{-1}$$
  \end{solucion}

  \begin{cuestion}
    Suponga la imagen de un plano en donde el vector $l=(l_1,l_2,l_3)$ representa
    la proyección de la recta del infinito del plano en la imagen. Sabemos que si
    conseguimos aplicar a nuestra imagen una homografía $G$ tal que si $l’= Gl$, siendo
    $l’^T =(0,0,1)$ entonces habremos rectificado nuestra imagen llevándola de nuevo
    al plano afín. Suponiendo que la recta definida por l no pasa por el punto $(0,0)$
    del plano imagen. Encontrar la homografía $G$. Justificar la respuesta

  \end{cuestion}

  \begin{solucion}
    ...
  \end{solucion}

  \begin{cuestion}
    Identificar los movimientos elementales (traslación, giro, escala,
    cizalla, proyectivo) representados por las homografías $H_1$, $H_2$, $H_3$ y $H_4$:

    \[
    H_1 =
    \left(
    \begin{array}{ccc}
        1 & 0 & 3 \\
        0 & 1 & 5 \\
        0 & 0 & 1 \\
    \end{array}
    \right)
    \left(
    \begin{array}{ccc}
        0.5 & 0 & 0 \\
        0 & 0.3 & 0 \\
        0 & 0 & 1 \\
    \end{array}
    \right)
    \left(
    \begin{array}{ccc}
        1 & 3 & 0 \\
        0 & 1 & 0 \\
        0 & 0 & 1 \\
    \end{array}
    \right)
    \]

    \[
    H_2 =
    \left(
    \begin{array}{ccc}
        0 & 1 & -3 \\
        -1 & 0 & 2 \\
        0 & 0 & 1 \\
    \end{array}
    \right)
    \left(
    \begin{array}{ccc}
        2 & 0 & 0 \\
        2 & 2 & 0 \\
        0 & 0 & 1 \\
    \end{array}
    \right)
    \]

    \[
    H_3 =
    \left(
    \begin{array}{ccc}
        1 & 0.5 & 0 \\
        0.5 & 2 & 0 \\
        0 & 0 & 1 \\
    \end{array}
    \right)
    \left(
    \begin{array}{ccc}
        1 & 0 & 0 \\
        0 & 1 & 0 \\
        -1 & 0 & 1 \\
    \end{array}
    \right)
    \]

    \[
    H_4 =
    \left(
    \begin{array}{ccc}
        2 & 0 & 3 \\
        0 & 2 & -1 \\
        0 & 1 & 2 \\
    \end{array}
    \right)
    \] \\

  \end{cuestion}

  \begin{solucion}
    Vamos a descomponer cada homografía en transformaciones conocidas. Si no digo
    lo contrario, la descomposición se hace con esta heurística: cojo 2 matrices de
    transformación (eligiendo las que parezcan más probables), las multiplico e intento
    ajustar contantes.
      \begin{itemize}
        \item $H_1$:
          Es claro que se trata de Traslación * Escala * Cizalla
        \item $H_2$:
          Descomponiendo ambas matrices queda Rotación * Traslación * Escala * Cizalla

          \[
          H_2 = \underbrace{
          \left(
          \begin{array}{ccc}
              0 & 1 & -3 \\
              -1 & 0 & 2 \\
              0 & 0 & 1 \\
          \end{array}
          \right)}_{A}
          \underbrace{
          \left(
          \begin{array}{ccc}
              2 & 0 & 0 \\
              2 & 2 & 0 \\
              0 & 0 & 1 \\
          \end{array}
          \right)}_{B} = \underbrace {\left(
                    \begin{array}{ccc}
                        0 & 1 & 0 \\
                        -1 & 0 & 0 \\
                        0 & 0 & 1 \\
                    \end{array}
                    \right)
                    \left(
                    \begin{array}{ccc}
                        1 & 0 & -2 \\
                        0 & 1 & -3 \\
                        0 & 0 & 1 \\
                    \end{array}
                    \right)}_{A}
                    \underbrace{
                    \left(
                    \begin{array}{ccc}
                        2 & 0 & 0 \\
                        0 & 2 & 0 \\
                        0 & 0 & 1 \\
                    \end{array}
                    \right)
                    \left(
                    \begin{array}{ccc}
                        1 & 0 & 0 \\
                        1 & 1 & 0 \\
                        0 & 0 & 1 \\
                    \end{array}
                    \right)}_{B}
          \]

        \item $H_3$:
          Descomponiendo queda Escala * Cizalla * Proyección
          \[
          H_3 = \underbrace{
          \left(
          \begin{array}{ccc}
              1 & 0.5 & 0 \\
              0.5 & 2 & 0 \\
              0 & 0 & 1 \\
          \end{array}
          \right)}_{A}
          \underbrace{
          \left(
          \begin{array}{ccc}
              1 & 0 & 0 \\
              0 & 1 & 0 \\
              -1 & 0 & 1 \\
          \end{array}
          \right)}_{B} = \underbrace{\left(
                    \begin{array}{ccc}
                        1 & 0 & 0 \\
                        0 & 2 & 0 \\
                        0 & 0 & 1 \\
                    \end{array}
                    \right)
                    \left(
                    \begin{array}{ccc}
                        1 & 0.5 & 0 \\
                        0.25 & 1 & 0 \\
                        0 & 0 & 1 \\
                    \end{array}
                    \right)}_{A}
                    \underbrace{
                    \left(
                    \begin{array}{ccc}
                        1 & 0 & 0 \\
                        0 & 1 & 0 \\
                        -1 & 0 & 1 \\
                    \end{array}
                    \right)}_{B}
          \]

        \item $H_4$: Pendiente
      \end{itemize}
  \end{solucion}

  \begin{cuestion}
    ¿Cuáles son las propiedades necesarias y suficientes para que una matriz
    defina una homografía entre planos? Justificar la respuesta
  \end{cuestion}

  \begin{solucion}
      Por \fnurl{definición}{https://en.wikipedia.org/wiki/Homography}, una homografía
      es un isomorfismo entre espacios proyectivos. Este isomorfismo viene inducido
      por un isomorfismo entre los espacios vectoriales de los que proviene dicho proyectivo.
      Por tanto, para que una matriz defina una homografía, debe inducir un isomorfismo
      entre espacios vectoriales, lo cual se cumple si y solamente si la matriz es invertible,
      que equivale a que su determinante sea distinto de cero. Así, cualquier matriz invertible
      definirá un isomorfismo entre espacios vectoriales, y por tanto un isomorfismo entre
      los espacios proyectivos inducidos.
  \end{solucion}

  \begin{cuestion}
    ¿Qué propiedades de la geometría de un plano quedan invariantes si se aplica
    una homografía general sobre él? Justificar la respuesta.
  \end{cuestion}

  \begin{solucion}
    Para una homografía en general, lo único que queda invariante serán las líneas
    rectas, que seguirán siendo rectas (3 puntos alineados seguirán estándolo tras la transformación).
    Si particularizamos, con una transformación afín conservamos también el paralelismo y las proporciones. Y si ahondamos aún más, con
    una aplicaen el 2 creemos que de la manera que lo has enfocado estás mezclando cosas de un espacio afín con otras del proyectivoción lineal conservamos lo anterior y además llevan el origen en el origen
    (no hay traslaciones).
  \end{solucion}

  \begin{cuestion}
    ¿Cuál es la deformación geométrica más fuerte que se puede producir sobre
    la imagen de un plano por el punto de vista de la cámara? Justificar la respuesta.
  \end{cuestion}

  \begin{solucion}
    La proyección, ya que lo único que conserva son las líneas rectas. Esta deformación
    produce, entre otras cosas, que las líneas paralelas ya no lo sean, produciendo
    deformaciones difíciles de reconstruir (aunque ya bastante estudiadas, ya que tienen muchos
    grados de libertad).
  \end{solucion}


  \begin{cuestion}
    ¿Qué información de la imagen usa el detector de Harris para seleccionar
    puntos? ¿El detector de Harris detecta patrones geométricos o fotométricos?
    Justificar la contestación.
  \end{cuestion}

  \begin{solucion}
     	Este detector utiliza la invarianza horizontal y vertical de la intensidad
      luminosa de los píxeles en busca de esquinas. Por tanto, lo que está detectando son patrones
      geométricos. Se basa en los valores propios de la matriz de error $SSD$ en busca
      de cambios bruscos de intensidad.
  \end{solucion}
  \begin{cuestion}
    ¿Sería adecuado usar como descriptor de un punto Harris los valores de
    los píxeles de su región de soporte? En caso positivo identificar cuando y
    justificar la respuesta
  \end{cuestion}

  \begin{solucion}
     	Generalmente no, ya que sería muy sensible frente a cualquier homografía
      que no sea una traslación.
  \end{solucion}
  \begin{cuestion}
    ¿Qué información de la imagen se codifica en el descriptor de SIFT?
    Justificar la contestación.
  \end{cuestion}

  \begin{solucion}
     	En el descriptor de SIFT se almacena un histograma de la orientación de los
      gradientes de cada píxel de una zona detectada. El histograma está formado
      por 8 direcciones del gradiente.
  \end{solucion}

  \begin{cuestion}
    Describa un par de criterios que sirvan para establecer correspondencias
    (matching) entre descriptores de regiones extraídos de dos imágenes. Justificar
    la idoneidad de los mismos
  \end{cuestion}

  \begin{solucion}
     	Para establecer correspondencias entre descriptores se define una función
      distancia que compara ambos descriptores. Se toma uno de la primera imagen
      y se comprueban todos los demás de la segunda para encontrar uno con distancia mínima.

      La primera aproximación a este hecho es tomar la distancia $L_2=\|f_1-f_2\|$,
      pero tiene el problema de que puede dar falsos positivos en regiones parecidas.
      Un ejemplo claro es en una estructura repetida (en los ejemplos de clase se ve
      muy fácil con una valla de jardín) donde las repeticiones en la imagen podrían
      llevar a confundir nuestro criterio.

      Un enfoque distinto sería tomar la distancia en proporción $d=\frac{\|f_1-f_2\|}{\|f_1-f_2'\|}$,
      ya que para correspondencias ambiguas da valores muy grandes, por lo que son
      desechados.
  \end{solucion}

  \begin{cuestion}
    Cual es el objetivo principal en el uso de la técnica RANSAC. Justificar
    la respuesta
  \end{cuestion}

  \begin{solucion}
     	El objetivo principal es dar una alternativa de función de coste a la de mínimos
      cuadrados. Esto quiere decir que, matemáticamente, busca realizar una regresión
      lineal entre un conjunto de puntos en el que la minimización por mínimos cuadrados
      no es suficiente para conseguir una buena homografía (que es lo que buscamos con
      dicha regresión). Esto lo consigue trazando líneas aleatorias entre las muestras
      y tomar la que mejor nos convenga según un criterio de satisfacibilidad de las muestras.

      En Visión por Computador se usa al estimar una homografía entre dos conjuntos
      de puntos. Así se desechan las correspondencias \textit{peores} entre ambos conjuntos y se consigue
      estimar una buena homografía.
  \end{solucion}
  \begin{cuestion}
    ¿Si tengo 4 imágenes de una escena de manera que se solapan la 1-2, 2-3
    y 3-4. ¿Cuál es el número mínimo de puntos en correspondencias necesarios para
    montar un mosaico? Justificar la respuesta
  \end{cuestion}

  \begin{solucion}
     	Para cada par de imágenes necesitamos como mínimo 4 imágenes. Este mínimo lo
      cumple el criterio RANSAC. Así que para 4 imágenes y 3 correspondencias, con
      4 puntos por correspondencia, necesitaremos 12 puntos en total como mínimo
      para montar un mosaico con dichas imágenes.
  \end{solucion}

  \begin{cuestion}
    En la confección de un mosaico con proyección rectangular es esperable
   que aparezcan deformaciones de la realidad. ¿Cuáles y porqué?.¿Bajo qué
   condiciones esas deformaciones podrían desaparecer? Justificar la respuesta

  \end{cuestion}

  \begin{solucion}
    Se esperan deformaciones de las imágenes más a los extremos. Esto es debido al
    \textit{error} que se va acumulando al realizar las proyecciones por homografías
    de coordenadas esféricas al plano. Para evitar estas deformaciones podemos usar
    una técnica común: empezar las proyecciones desde la imagen central y llegar
    hasta ambos extremos. Así las homografías no son tan fuertes y deforman menos la imagen.
    Pero en el contexto de que haya muchas imágenes con un cambio fuerte de coordenadas
    esféricas, la deformación será más clara y cada vez será más difícil realizar
    un mosaico con poca deformación.
  \end{solucion}


\end{document}
