\documentclass[a4paper, 11pt]{article}

%Comandos para configurar el idioma
\usepackage[spanish,activeacute]{babel}
\usepackage[utf8]{inputenc}
\usepackage[T1]{fontenc} %Necesario para el uso de las comillas latinas.
\usepackage{geometry} % Used to adjust the document margins

%Importante que esta sea la última órden del preámbulo
\usepackage{hyperref}
   \hypersetup{
     pdftitle={Cuestionario de teoría 3},
     pdfauthor={Antonio Álvarez Caballero},
     unicode,
     breaklinks=true,  % so long urls are correctly broken across lines
     colorlinks=true,
     urlcolor=blue,
     linkcolor=darkorange,
     citecolor=darkgreen,
     }

   % Slightly bigger margins than the latex defaults

   \geometry{verbose,tmargin=1in,bmargin=1in,lmargin=1in,rmargin=1in}
\newcommand\fnurl[2]{%
  \href{#2}{#1}\footnote{\url{#2}}%
}

%Paquetes matemáticos
\usepackage{amsmath,amsfonts,amsthm}
\usepackage[all]{xy} %Para diagramas
\usepackage{enumerate} %Personalización de enumeraciones
\usepackage{tikz} %Dibujos
\usepackage{ wasysym } % Para la sonrisa

%Tipografía escalable
\usepackage{lmodern}
%Legibilidad
\usepackage{microtype}

%Código
\usepackage{listings}
\usepackage{color}

\definecolor{dkgreen}{rgb}{0,0.6,0}
\definecolor{gray}{rgb}{0.5,0.5,0.5}
\definecolor{mauve}{rgb}{0.58,0,0.82}
\definecolor{orange}{cmyk}{0,0.4,0.8,0.2}
\definecolor{darkorange}{rgb}{.71,0.21,0.01}
\definecolor{darkgreen}{rgb}{.12,.54,.11}
\definecolor{myteal}{rgb}{.26, .44, .56}
\definecolor{gray}{gray}{0.45}
\definecolor{lightgray}{gray}{.95}
\definecolor{mediumgray}{gray}{.8}

\setlength{\parskip}{.5em} % por defecto el espacio entre párrafos es 0pt




\lstset{frame=tb,
  language=Python,
  aboveskip=3mm,
  belowskip=3mm,
  showstringspaces=false,
  columns=flexible,
  basicstyle={\small\ttfamily},
  numbers=left,
  numberstyle=\tiny\color{gray},
  keywordstyle=\color{blue},
  commentstyle=\color{dkgreen},
  stringstyle=\color{mauve},
  breaklines=true,
  breakatwhitespace=true,
  tabsize=3
}

\title{Cuestionario de teoría 3}
\author{Antonio Álvarez Caballero\\
    \href{mailto:analca3@correo.ugr.es}{analca3@correo.ugr.es}}
\date{\today}

\theoremstyle{definition}
\newtheorem{ejercicio}{Ejercicio}
\newtheorem{cuestion}{Cuestión}
\newtheorem*{solucion}{Solución}
\newtheorem*{bonus}{BONUS}


%%%%%%%% New sqrt
\usepackage{letltxmacro}
\makeatletter
\let\oldr@@t\r@@t
\def\r@@t#1#2{%
\setbox0=\hbox{$\oldr@@t#1{#2\,}$}\dimen0=\ht0
\advance\dimen0-0.2\ht0
\setbox2=\hbox{\vrule height\ht0 depth -\dimen0}%
{\box0\lower0.4pt\box2}}
\LetLtxMacro{\oldsqrt}{\sqrt}
\renewcommand*{\sqrt}[2][\ ]{\oldsqrt[#1]{#2} }
\makeatother

%%%%%%%%%%%%%%%%%%%%%%%%%%%%%%%%%%%%%%%%%%%%%

\begin{document}

  \maketitle

  \section{Cuestiones}

  \begin{cuestion}
    En clase se ha mostrado una técnica para estimar el vector de traslación del movimiento de
    un par estéreo y solo ha podido estimarse su dirección. Argumentar de forma lógica a favor o
    en contra del hecho de que dicha restricción sea debida a la técnica usada o sea un problema
    inherente a la reconstrucción.
  \end{cuestion}

  \begin{solucion}
    Esta restricción es debida a la técnica usada. Dicho método estima la matriz
    esencial utilizando esta igualdad:
    \[ E^T E = [T]_x R R^T [T] = [T]_x [T] = \left(
    \begin{array}{ccc}
        T_y^2 + T_x^2 & -T_xT_y & -T_xT_z \\
        -T_yT_x & T_z^2 + T_x^2 & -T_yT_z \\
        -T_zT_x & -T_zT_y & T_z^2 + T_x^2 \\
    \end{array}
    \right) \]
    Aún normalizando la matriz con su traza, siguen apareciendo términos cuadráticos,
    lo que al ir resolviendo el sistema $3\times3$ que sale (tomando cualquier fila o columna de la normalizada)
    dichos términos presentan ambigüedad en su signo al tomar raíces cuadradas, por
    lo que se toma uno cualquiera y después en el \textit{Algoritmo de Reconstrucción Euclídea}
    se corrige. Por el hecho de que se corrige podemos afirmar que no es algo inherente a la reconstrucción.
  \end{solucion}

  \begin{cuestion}
    Verificar matemáticamente que se deben de cumplir las ecuaciones $Fe = 0, F^Te'= 0$

  \end{cuestion}

  \begin{solucion} % Pag 245 (iii)
    Según \emph{Hartley \& Zisselman}, para cada punto $x$ distinto de $e$, la línea
    epipolar $l'=Fx$ contiene el epipolo $e'$. Este epipolo cumple $e'^T(Fx)=(e'^TF)x=0$
    para cualquier $x$. Esto sólo puede darse si $e'^TF=0$, que significa que $e'$
    anula a $F^T$ por la izquierda. Análogamente para $e$, se determina que $Fe=0$,
    anulando $e$ a $F$ por la derecha.
  \end{solucion}

  \begin{cuestion}
    Verificar matemáticamente que cuando una cámara se desplaza las coordenadas retina de
puntos correspondientes sobre la retina deben de verificar la ecuación $ x' = x + \frac{Kt}{Z} $

  \end{cuestion}

  \begin{solucion}
      ...
  \end{solucion}

  \begin{cuestion}
    Dar una interpretación geométrica a las columnas y filas de la matriz $P$ de una cámara.
  \end{cuestion}

  \begin{solucion} % Pag 158 Tabla 6.1
    Según \emph{Hartley \& Zisselman}, las columnas $P_i$ de la matriz de una cámara son
    los puntos muertos de la imagen en cada uno de los ejes $X$, $Y$, y $Z$. La última fila
    es la imagen del origen de coordenadas.

    Las filas $1$ y $2$ representan planos en el espacio $x=0$ e $y=0$ que claramente
    cortan el centro de la cámara. La fila $3$ representa el plano principal de la cámara.
  \end{solucion}

  \begin{cuestion}
    Suponga una matriz $ A \in \mathcal{M}_{3\times3}$ de números reales.
    Suponga $rango(A)=3$. ¿Cuál es la matriz esencial más cercana a A en norma de Frobenius?
    Argumentar matemáticamente la derivación.
  \end{cuestion}

  \begin{solucion}
    ...
  \end{solucion}

  \begin{cuestion}
    Discutir cuales son las ventajas y desventajas de usar un algoritmo de reconstrucción
    Euclídea que calcule la profundidad de varios puntos a la vez en lugar de uno a uno.

  \end{cuestion}

  \begin{solucion}
      ...
  \end{solucion}

  \begin{cuestion}
    Deducir la expresión para la matriz Esencial $E = [t]_x R = R[R^Tt]_x$.
    Justificar cada uno de los pasos

  \end{cuestion}

  \begin{solucion}
    ...
  \end{solucion}

  \begin{cuestion}
    Dada una pareja de cámaras cualesquiera, ¿existen puntos del espacio que no tengan un
    plano epipolar asociado? Justificar la respuesta
  \end{cuestion}

  \begin{solucion}
    Reduzcamos este problema a uno de geometría en $\mathcal{R}^3$: Para un
    punto $x$ fijo (punto escena) y dos puntos $p,q$ libres en el
    espacio (las cámaras), ¿existe siempre un plano (plano epipolar) entre estos 3 puntos?
    La respuesta es clara: no. Si los 3 puntos forman una recta, no pueden formar un plano.
    Estos 3 puntos deben ser linealmente independientes, formar una base. Es un problema
    básico de geomtría en el espacio euclídeo.
  \end{solucion}


  \begin{cuestion}
    Si nos dan las coordenadas de proyección de un punto escena en la cámara-1 y nos dicen
    cuál es el movimiento relativo de la cámara-2 respecto de la cámara-1, ¿es posible reconstruir
    la profundidad del punto si las cámaras están calibradas?. Justificar la contestación
  \end{cuestion}

  \begin{solucion}
     	No, porque teniendo sólo el punto de escena proyectado en una cámara, aún estando
      calibradas ambas, no podemos conseguir el punto escena proyectado en la otra. Al menos
      necesitamos las proyecciones del punto escena en ambas cámaras para establecer correspondencias. 
  \end{solucion}


  \begin{cuestion}
    Suponga que obtiene una foto de una escena y la cámara gira para obtener otra foto ¿Cuál
    es la ecuación que liga las coordenadas de la proyecciones en ambas imágenes, de un mismo
    punto escena, en términos de los parámetros de las cámaras. Justificar matemáticamente

  \end{cuestion}

  \begin{solucion}
     	...
  \end{solucion}
  \begin{cuestion}
    Suponga una cámara Afín. Discutir cuales son los efectos de la proyección ortogonal sobre
    los parámetros intrínsecos y extrínsecos de la cámara.

  \end{cuestion}

  \begin{solucion}
     	...
  \end{solucion}

  \begin{cuestion}
    Dadas dos cámaras calibradas, P=K[I|0] y P'=K'[R|t]. Calcular la expresión de la matriz
    fundamental en términos de los parámetros intrínsecos y extrínsecos de las cámaras. Todos los
    pasos deben ser justificados

  \end{cuestion}

  \begin{solucion}
     	...
  \end{solucion}


\end{document}
