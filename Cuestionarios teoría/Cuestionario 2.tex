\documentclass[a4paper, 11pt]{article}

%Comandos para configurar el idioma
\usepackage[spanish,activeacute]{babel}
\usepackage[utf8]{inputenc}
\usepackage[T1]{fontenc} %Necesario para el uso de las comillas latinas.
\usepackage{geometry} % Used to adjust the document margins

%Importante que esta sea la última órden del preámbulo
\usepackage{hyperref}
   \hypersetup{
     pdftitle={Cuestionario de teoría 2},
     pdfauthor={Antonio Álvarez Caballero},
     unicode,
     breaklinks=true,  % so long urls are correctly broken across lines
     colorlinks=true,
     urlcolor=blue,
     linkcolor=darkorange,
     citecolor=darkgreen,
     }

   % Slightly bigger margins than the latex defaults

   \geometry{verbose,tmargin=1in,bmargin=1in,lmargin=1in,rmargin=1in}
\newcommand\fnurl[2]{%
  \href{#2}{#1}\footnote{\url{#2}}%
}

%Paquetes matemáticos
\usepackage{amsmath,amsfonts,amsthm}
\usepackage[all]{xy} %Para diagramas
\usepackage{enumerate} %Personalización de enumeraciones
\usepackage{tikz} %Dibujos

%Tipografía escalable
\usepackage{lmodern}
%Legibilidad
\usepackage{microtype}

%Código
\usepackage{listings}
\usepackage{color}

\definecolor{dkgreen}{rgb}{0,0.6,0}
\definecolor{gray}{rgb}{0.5,0.5,0.5}
\definecolor{mauve}{rgb}{0.58,0,0.82}

\lstset{frame=tb,
  language=Python,
  aboveskip=3mm,
  belowskip=3mm,
  showstringspaces=false,
  columns=flexible,
  basicstyle={\small\ttfamily},
  numbers=left,
  numberstyle=\tiny\color{gray},
  keywordstyle=\color{blue},
  commentstyle=\color{dkgreen},
  stringstyle=\color{mauve},
  breaklines=true,
  breakatwhitespace=true,
  tabsize=3
}

\title{Cuestionario de teoría 2}
\author{Antonio Álvarez Caballero\\
    \href{mailto:analca3@correo.ugr.es}{analca3@correo.ugr.es}}
\date{\today}

\theoremstyle{definition}
\newtheorem{ejercicio}{Ejercicio}
\newtheorem{cuestion}{Cuestión}
\newtheorem*{solucion}{Solución}
\newtheorem*{bonus}{BONUS}

% %%%%%%%%%%%%%%%%%%%%%%%%%% IPNB
% \usepackage{graphicx} % Used to insert images
%    \usepackage{adjustbox} % Used to constrain images to a maximum size
%    \usepackage{color} % Allow colors to be defined
%    \usepackage{enumerate} % Needed for markdown enumerations to work
%    \usepackage{geometry} % Used to adjust the document margins
%    \usepackage{amsmath} % Equations
%    \usepackage{amssymb} % Equations
%    \usepackage{eurosym} % defines \euro
%    \usepackage[mathletters]{ucs} % Extended unicode (utf-8) support
%    \usepackage[utf8x]{inputenc} % Allow utf-8 characters in the tex document
%   % \usepackage{fancyvrb} % verbatim replacement that allows latex
%    \usepackage{grffile} % extends the file name processing of package graphics
%                         % to support a larger range
%    % The hyperref package gives us a pdf with properly built
%    % internal navigation ('pdf bookmarks' for the table of contents,
%    % internal cross-reference links, web links for URLs, etc.)
%    \usepackage{hyperref}
%    \usepackage{longtable} % longtable support required by pandoc >1.10
%    \usepackage{booktabs}  % table support for pandoc > 1.12.2
%
%
%
%
%    \definecolor{orange}{cmyk}{0,0.4,0.8,0.2}
%    \definecolor{darkorange}{rgb}{.71,0.21,0.01}
%    \definecolor{darkgreen}{rgb}{.12,.54,.11}
%    \definecolor{myteal}{rgb}{.26, .44, .56}
%    \definecolor{gray}{gray}{0.45}
%    \definecolor{lightgray}{gray}{.95}
%    \definecolor{mediumgray}{gray}{.8}
%    \definecolor{inputbackground}{rgb}{.95, .95, .85}
%    \definecolor{outputbackground}{rgb}{.95, .95, .95}
%    \definecolor{traceback}{rgb}{1, .95, .95}
%    % ansi colors
%    \definecolor{red}{rgb}{.6,0,0}
%    \definecolor{green}{rgb}{0,.65,0}
%    \definecolor{brown}{rgb}{0.6,0.6,0}
%    \definecolor{blue}{rgb}{0,.145,.698}
%    \definecolor{purple}{rgb}{.698,.145,.698}
%    \definecolor{cyan}{rgb}{0,.698,.698}
%    \definecolor{lightgray}{gray}{0.5}
%
%    % bright ansi colors
%    \definecolor{darkgray}{gray}{0.25}
%    \definecolor{lightred}{rgb}{1.0,0.39,0.28}
%    \definecolor{lightgreen}{rgb}{0.48,0.99,0.0}
%    \definecolor{lightblue}{rgb}{0.53,0.81,0.92}
%    \definecolor{lightpurple}{rgb}{0.87,0.63,0.87}
%    \definecolor{lightcyan}{rgb}{0.5,1.0,0.83}
%
%    % commands and environments needed by pandoc snippets
%    % extracted from the output of `pandoc -s`
%    \providecommand{\tightlist}{%
%      \setlength{\itemsep}{0pt}\setlength{\parskip}{0pt}}
%    \DefineVerbatimEnvironment{Highlighting}{Verbatim}{commandchars=\\\{\}}
%    % Add ',fontsize=\small' for more characters per line
%    \newenvironment{Shaded}{}{}
%    \newcommand{\KeywordTok}[1]{\textcolor[rgb]{0.00,0.44,0.13}{\textbf{{#1}}}}
%    \newcommand{\DataTypeTok}[1]{\textcolor[rgb]{0.56,0.13,0.00}{{#1}}}
%    \newcommand{\DecValTok}[1]{\textcolor[rgb]{0.25,0.63,0.44}{{#1}}}
%    \newcommand{\BaseNTok}[1]{\textcolor[rgb]{0.25,0.63,0.44}{{#1}}}
%    \newcommand{\FloatTok}[1]{\textcolor[rgb]{0.25,0.63,0.44}{{#1}}}
%    \newcommand{\CharTok}[1]{\textcolor[rgb]{0.25,0.44,0.63}{{#1}}}
%    \newcommand{\StringTok}[1]{\textcolor[rgb]{0.25,0.44,0.63}{{#1}}}
%    \newcommand{\CommentTok}[1]{\textcolor[rgb]{0.38,0.63,0.69}{\textit{{#1}}}}
%    \newcommand{\OtherTok}[1]{\textcolor[rgb]{0.00,0.44,0.13}{{#1}}}
%    \newcommand{\AlertTok}[1]{\textcolor[rgb]{1.00,0.00,0.00}{\textbf{{#1}}}}
%    \newcommand{\FunctionTok}[1]{\textcolor[rgb]{0.02,0.16,0.49}{{#1}}}
%    \newcommand{\RegionMarkerTok}[1]{{#1}}
%    \newcommand{\ErrorTok}[1]{\textcolor[rgb]{1.00,0.00,0.00}{\textbf{{#1}}}}
%    \newcommand{\NormalTok}[1]{{#1}}
%
%    % Define a nice break command that doesn't care if a line doesn't already
%    % exist.
%    \def\br{\hspace*{\fill} \\* }
%    % Math Jax compatability definitions
%    \def\gt{>}
%    \def\lt{<}
%    % Document parameters
%    \title{Cuestionario de teoría 2}
%    \author{Antonio Álvarez Caballero\\
%        \href{mailto:analca3@correo.ugr.es}{analca3@correo.ugr.es}}
%    \date{\today}
%
%
%
%
%    % Pygments definitions
%
% \makeatletter
% \def\PY@reset{\let\PY@it=\relax \let\PY@bf=\relax%
%    \let\PY@ul=\relax \let\PY@tc=\relax%
%    \let\PY@bc=\relax \let\PY@ff=\relax}
% \def\PY@tok#1{\csname PY@tok@#1\endcsname}
% \def\PY@toks#1+{\ifx\relax#1\empty\else%
%    \PY@tok{#1}\expandafter\PY@toks\fi}
% \def\PY@do#1{\PY@bc{\PY@tc{\PY@ul{%
%    \PY@it{\PY@bf{\PY@ff{#1}}}}}}}
% \def\PY#1#2{\PY@reset\PY@toks#1+\relax+\PY@do{#2}}
%
% \expandafter\def\csname PY@tok@gr\endcsname{\def\PY@tc##1{\textcolor[rgb]{1.00,0.00,0.00}{##1}}}
% \expandafter\def\csname PY@tok@gh\endcsname{\let\PY@bf=\textbf\def\PY@tc##1{\textcolor[rgb]{0.00,0.00,0.50}{##1}}}
% \expandafter\def\csname PY@tok@gu\endcsname{\let\PY@bf=\textbf\def\PY@tc##1{\textcolor[rgb]{0.50,0.00,0.50}{##1}}}
% \expandafter\def\csname PY@tok@ge\endcsname{\let\PY@it=\textit}
% \expandafter\def\csname PY@tok@kp\endcsname{\def\PY@tc##1{\textcolor[rgb]{0.00,0.50,0.00}{##1}}}
% \expandafter\def\csname PY@tok@go\endcsname{\def\PY@tc##1{\textcolor[rgb]{0.53,0.53,0.53}{##1}}}
% \expandafter\def\csname PY@tok@sc\endcsname{\def\PY@tc##1{\textcolor[rgb]{0.73,0.13,0.13}{##1}}}
% \expandafter\def\csname PY@tok@o\endcsname{\def\PY@tc##1{\textcolor[rgb]{0.40,0.40,0.40}{##1}}}
% \expandafter\def\csname PY@tok@nd\endcsname{\def\PY@tc##1{\textcolor[rgb]{0.67,0.13,1.00}{##1}}}
% \expandafter\def\csname PY@tok@ni\endcsname{\let\PY@bf=\textbf\def\PY@tc##1{\textcolor[rgb]{0.60,0.60,0.60}{##1}}}
% \expandafter\def\csname PY@tok@se\endcsname{\let\PY@bf=\textbf\def\PY@tc##1{\textcolor[rgb]{0.73,0.40,0.13}{##1}}}
% \expandafter\def\csname PY@tok@vi\endcsname{\def\PY@tc##1{\textcolor[rgb]{0.10,0.09,0.49}{##1}}}
% \expandafter\def\csname PY@tok@c1\endcsname{\let\PY@it=\textit\def\PY@tc##1{\textcolor[rgb]{0.25,0.50,0.50}{##1}}}
% \expandafter\def\csname PY@tok@bp\endcsname{\def\PY@tc##1{\textcolor[rgb]{0.00,0.50,0.00}{##1}}}
% \expandafter\def\csname PY@tok@na\endcsname{\def\PY@tc##1{\textcolor[rgb]{0.49,0.56,0.16}{##1}}}
% \expandafter\def\csname PY@tok@s1\endcsname{\def\PY@tc##1{\textcolor[rgb]{0.73,0.13,0.13}{##1}}}
% \expandafter\def\csname PY@tok@no\endcsname{\def\PY@tc##1{\textcolor[rgb]{0.53,0.00,0.00}{##1}}}
% \expandafter\def\csname PY@tok@gt\endcsname{\def\PY@tc##1{\textcolor[rgb]{0.00,0.27,0.87}{##1}}}
% \expandafter\def\csname PY@tok@kt\endcsname{\def\PY@tc##1{\textcolor[rgb]{0.69,0.00,0.25}{##1}}}
% \expandafter\def\csname PY@tok@c\endcsname{\let\PY@it=\textit\def\PY@tc##1{\textcolor[rgb]{0.25,0.50,0.50}{##1}}}
% \expandafter\def\csname PY@tok@nl\endcsname{\def\PY@tc##1{\textcolor[rgb]{0.63,0.63,0.00}{##1}}}
% \expandafter\def\csname PY@tok@s2\endcsname{\def\PY@tc##1{\textcolor[rgb]{0.73,0.13,0.13}{##1}}}
% \expandafter\def\csname PY@tok@gd\endcsname{\def\PY@tc##1{\textcolor[rgb]{0.63,0.00,0.00}{##1}}}
% \expandafter\def\csname PY@tok@vc\endcsname{\def\PY@tc##1{\textcolor[rgb]{0.10,0.09,0.49}{##1}}}
% \expandafter\def\csname PY@tok@mo\endcsname{\def\PY@tc##1{\textcolor[rgb]{0.40,0.40,0.40}{##1}}}
% \expandafter\def\csname PY@tok@cm\endcsname{\let\PY@it=\textit\def\PY@tc##1{\textcolor[rgb]{0.25,0.50,0.50}{##1}}}
% \expandafter\def\csname PY@tok@sb\endcsname{\def\PY@tc##1{\textcolor[rgb]{0.73,0.13,0.13}{##1}}}
% \expandafter\def\csname PY@tok@kd\endcsname{\let\PY@bf=\textbf\def\PY@tc##1{\textcolor[rgb]{0.00,0.50,0.00}{##1}}}
% \expandafter\def\csname PY@tok@nn\endcsname{\let\PY@bf=\textbf\def\PY@tc##1{\textcolor[rgb]{0.00,0.00,1.00}{##1}}}
% \expandafter\def\csname PY@tok@cp\endcsname{\def\PY@tc##1{\textcolor[rgb]{0.74,0.48,0.00}{##1}}}
% \expandafter\def\csname PY@tok@sx\endcsname{\def\PY@tc##1{\textcolor[rgb]{0.00,0.50,0.00}{##1}}}
% \expandafter\def\csname PY@tok@nb\endcsname{\def\PY@tc##1{\textcolor[rgb]{0.00,0.50,0.00}{##1}}}
% \expandafter\def\csname PY@tok@gi\endcsname{\def\PY@tc##1{\textcolor[rgb]{0.00,0.63,0.00}{##1}}}
% \expandafter\def\csname PY@tok@kr\endcsname{\let\PY@bf=\textbf\def\PY@tc##1{\textcolor[rgb]{0.00,0.50,0.00}{##1}}}
% \expandafter\def\csname PY@tok@vg\endcsname{\def\PY@tc##1{\textcolor[rgb]{0.10,0.09,0.49}{##1}}}
% \expandafter\def\csname PY@tok@kc\endcsname{\let\PY@bf=\textbf\def\PY@tc##1{\textcolor[rgb]{0.00,0.50,0.00}{##1}}}
% \expandafter\def\csname PY@tok@gp\endcsname{\let\PY@bf=\textbf\def\PY@tc##1{\textcolor[rgb]{0.00,0.00,0.50}{##1}}}
% \expandafter\def\csname PY@tok@w\endcsname{\def\PY@tc##1{\textcolor[rgb]{0.73,0.73,0.73}{##1}}}
% \expandafter\def\csname PY@tok@gs\endcsname{\let\PY@bf=\textbf}
% \expandafter\def\csname PY@tok@mf\endcsname{\def\PY@tc##1{\textcolor[rgb]{0.40,0.40,0.40}{##1}}}
% \expandafter\def\csname PY@tok@il\endcsname{\def\PY@tc##1{\textcolor[rgb]{0.40,0.40,0.40}{##1}}}
% \expandafter\def\csname PY@tok@mi\endcsname{\def\PY@tc##1{\textcolor[rgb]{0.40,0.40,0.40}{##1}}}
% \expandafter\def\csname PY@tok@mh\endcsname{\def\PY@tc##1{\textcolor[rgb]{0.40,0.40,0.40}{##1}}}
% \expandafter\def\csname PY@tok@m\endcsname{\def\PY@tc##1{\textcolor[rgb]{0.40,0.40,0.40}{##1}}}
% \expandafter\def\csname PY@tok@sh\endcsname{\def\PY@tc##1{\textcolor[rgb]{0.73,0.13,0.13}{##1}}}
% \expandafter\def\csname PY@tok@sd\endcsname{\let\PY@it=\textit\def\PY@tc##1{\textcolor[rgb]{0.73,0.13,0.13}{##1}}}
% \expandafter\def\csname PY@tok@nv\endcsname{\def\PY@tc##1{\textcolor[rgb]{0.10,0.09,0.49}{##1}}}
% \expandafter\def\csname PY@tok@nt\endcsname{\let\PY@bf=\textbf\def\PY@tc##1{\textcolor[rgb]{0.00,0.50,0.00}{##1}}}
% \expandafter\def\csname PY@tok@ss\endcsname{\def\PY@tc##1{\textcolor[rgb]{0.10,0.09,0.49}{##1}}}
% \expandafter\def\csname PY@tok@mb\endcsname{\def\PY@tc##1{\textcolor[rgb]{0.40,0.40,0.40}{##1}}}
% \expandafter\def\csname PY@tok@nf\endcsname{\def\PY@tc##1{\textcolor[rgb]{0.00,0.00,1.00}{##1}}}
% \expandafter\def\csname PY@tok@ne\endcsname{\let\PY@bf=\textbf\def\PY@tc##1{\textcolor[rgb]{0.82,0.25,0.23}{##1}}}
% \expandafter\def\csname PY@tok@s\endcsname{\def\PY@tc##1{\textcolor[rgb]{0.73,0.13,0.13}{##1}}}
% \expandafter\def\csname PY@tok@err\endcsname{\def\PY@bc##1{\setlength{\fboxsep}{0pt}\fcolorbox[rgb]{1.00,0.00,0.00}{1,1,1}{\strut ##1}}}
% \expandafter\def\csname PY@tok@kn\endcsname{\let\PY@bf=\textbf\def\PY@tc##1{\textcolor[rgb]{0.00,0.50,0.00}{##1}}}
% \expandafter\def\csname PY@tok@cs\endcsname{\let\PY@it=\textit\def\PY@tc##1{\textcolor[rgb]{0.25,0.50,0.50}{##1}}}
% \expandafter\def\csname PY@tok@ow\endcsname{\let\PY@bf=\textbf\def\PY@tc##1{\textcolor[rgb]{0.67,0.13,1.00}{##1}}}
% \expandafter\def\csname PY@tok@k\endcsname{\let\PY@bf=\textbf\def\PY@tc##1{\textcolor[rgb]{0.00,0.50,0.00}{##1}}}
% \expandafter\def\csname PY@tok@nc\endcsname{\let\PY@bf=\textbf\def\PY@tc##1{\textcolor[rgb]{0.00,0.00,1.00}{##1}}}
% \expandafter\def\csname PY@tok@si\endcsname{\let\PY@bf=\textbf\def\PY@tc##1{\textcolor[rgb]{0.73,0.40,0.53}{##1}}}
% \expandafter\def\csname PY@tok@sr\endcsname{\def\PY@tc##1{\textcolor[rgb]{0.73,0.40,0.53}{##1}}}
%
% \def\PYZbs{\char`\\}
% \def\PYZus{\char`\_}
% \def\PYZob{\char`\{}
% \def\PYZcb{\char`\}}
% \def\PYZca{\char`\^}
% \def\PYZam{\char`\&}
% \def\PYZlt{\char`\<}
% \def\PYZgt{\char`\>}
% \def\PYZsh{\char`\#}
% \def\PYZpc{\char`\%}
% \def\PYZdl{\char`\$}
% \def\PYZhy{\char`\-}
% \def\PYZsq{\char`\'}
% \def\PYZdq{\char`\"}
% \def\PYZti{\char`\~}
% % for compatibility with earlier versions
% \def\PYZat{@}
% \def\PYZlb{[}
% \def\PYZrb{]}
% \makeatother
%
%
%    % Exact colors from NB
%    \definecolor{incolor}{rgb}{0.0, 0.0, 0.5}
%    \definecolor{outcolor}{rgb}{0.545, 0.0, 0.0}
%
%
%
%
%    % Prevent overflowing lines due to hard-to-break entities
%    \sloppy
%    % Setup hyperref package
%   %  \hypersetup{
%   %    pdftitle={Cuestionario de teoría 1},
%   %    pdfauthor={Antonio Álvarez Caballero},
%   %    unicode,
%   %    plainpages=false,
%   %    colorlinks,
%   %    citecolor=black,
%   %    filecolor=black,
%   %    linkcolor=black,
%   %    urlcolor=black,
%   %  }
%    \hypersetup{
%      pdftitle={Cuestionario de teoría 1},
%      pdfauthor={Antonio Álvarez Caballero},
%      unicode,
%      breaklinks=true,  % so long urls are correctly broken across lines
%      colorlinks=true,
%      urlcolor=blue,
%      linkcolor=darkorange,
%      citecolor=darkgreen,
%      }
%    % Slightly bigger margins than the latex defaults
%
%    \geometry{verbose,tmargin=1in,bmargin=1in,lmargin=1in,rmargin=1in}
%
%
%
%
%


%%%%%%%%%%%%%%%%%%%%%%%%%%%%%%%%%%%%%%%%



%%%%%%%% New sqrt
\usepackage{letltxmacro}
\makeatletter
\let\oldr@@t\r@@t
\def\r@@t#1#2{%
\setbox0=\hbox{$\oldr@@t#1{#2\,}$}\dimen0=\ht0
\advance\dimen0-0.2\ht0
\setbox2=\hbox{\vrule height\ht0 depth -\dimen0}%
{\box0\lower0.4pt\box2}}
\LetLtxMacro{\oldsqrt}{\sqrt}
\renewcommand*{\sqrt}[2][\ ]{\oldsqrt[#1]{#2} }
\makeatother

%%%%%%%%%%%%%%%%%%%%%%%%%%%%%%%%%%%%%%%%%%%%%

\begin{document}

  \maketitle

  \section{Cuestiones}

  \begin{cuestion}
    Identificar la/s diferencia/s esencial/es entre el plano afín y el plano
    proyectivo. ¿Cuáles son sus consecuencias? Justificar la contestación.

  \end{cuestion}

  \begin{solucion}
    cosas
  \end{solucion}

  \begin{cuestion}
    Verificar que en coordenadas homogéneas el vector de la recta definida por
    dos puntos puede calcularse como el producto vectorial de los vectores de los
    puntos $(l = x \times x’)$. De igual modo el punto intersección de dos rectas $l$ y
    $l’$ está dado por $x = l \times l’$

  \end{cuestion}

  \begin{solucion}
      cosas
  \end{solucion}

  \begin{cuestion}
    Sean $x$ y $l$ un punto y una recta respectivamente en un plano proyectivo $P1$
    y suponemos que la recta $l$ pasa por el punto $x$, es decir $l^Tx=0$. Sean $x'$ y $l'$
    un punto y una recta del plano proyectivo $P'$ donde al igual que antes $(l')^T x'=0$.
    Supongamos que existe un homografía de puntos $H$ entre ambos planos proyectivos,
    es decir $x'=Hx$. Deducir de las ecuaciones anteriores la expresión para la
    homografía $G$ que relaciona los vectores de las rectas, es decir $G$ tal que $l'=Gl$.
    Justificar la respuesta

  \end{cuestion}

  \begin{solucion}
      Partimos de
      $$(l')^T x' = 0$$
      Usando la homografía $H$ deducimos que
      $$(l')^T Hx = 0$$
      Ahora reagrupamos y lo escribimos como
      $$\big((l')^T H\big)x = 0$$
      Como sabemos que $l^T x = 0$  deducimos
      $$\big((l')^T H\big) = l^T$$
      Trasponemos ambos lados de la ecuación (la traspuesta invierte el orden del producto)
      $$H^T l' = l$$
      Ahora multiplicamos por la inversa de $H^T$ por la izquierda en ambos miembros
      $$l' = (H^T)^{-1} l$$
      Por tanto nuestra conclusión es
      $$G = (H^T)^{-1}$$
  \end{solucion}

  \begin{cuestion}
    Suponga la imagen de un plano en donde el vector $l=(l_1,l_2,l_3)$ representa
    la proyección de la recta del infinito del plano en la imagen. Sabemos que si
    conseguimos aplicar a nuestra imagen una homografía $G$ tal que si $l’= Gl$, siendo
    $l’^T =(0,0,1)$ entonces habremos rectificado nuestra imagen llevándola de nuevo
    al plano afín. Suponiendo que la recta definida por l no pasa por el punto $(0,0)$
    del plano imagen. Encontrar la homografía $G$. Justificar la respuesta

  \end{cuestion}

  \begin{solucion}
    cosas
  \end{solucion}

  \begin{cuestion}
    Identificar los movimientos elementales (traslación, giro, escala,
    cizalla, proyectivo) representados por las homografías $H1$, $H2$, $H3$ y $H4$:

  \end{cuestion}

  \begin{solucion}
      \begin{itemize}
        \item $H_1$
          Es claro que se trata de Traslación * Escala * Cizalla
        \item $H_2$
          Descomponemos ambas matrices (insertar) y nos queda Rotación(-90º) * Traslación * Escala * Cizalla
        \item $H_3$

        \item $H_4$:
  \end{solucion}

  \begin{cuestion}
    ¿Cuáles son las propiedades necesarias y suficientes para que una matriz
    defina una homografía entre planos? Justificar la respuesta
  \end{cuestion}

  \begin{solucion}
      cosas
  \end{solucion}

  \begin{cuestion}
    ¿Qué propiedades de la geometría de un plano quedan invariantes si se aplica
    una homografía general sobre él? Justificar la respuesta.
  \end{cuestion}

  \begin{solucion}
    cosas
  \end{solucion}

  \begin{cuestion}
    ¿Cuál es la deformación geométrica más fuerte que se puede producir sobre
    la imagen de un plano por el punto de vista de la cámara? Justificar la respuesta.
  \end{cuestion}

  \begin{solucion}

  \end{solucion}


  \begin{cuestion}
    ¿Qué información de la imagen usa el detector de Harris para seleccionar
    puntos? ¿El detector de Harris detecta patrones geométricos o fotométricos?
    Justificar la contestación.
  \end{cuestion}

  \begin{solucion}
     	Este detector utiliza la invarianza horizontal y vertical de la intensidad
      luminosa de los píxeles. Por tanto, lo que está detectando son patrones
      fotométricos.
  \end{solucion}
  \begin{cuestion}
    ¿Sería adecuado usar como descriptor de un punto Harris los valores de
    los píxeles de su región de soporte? En caso positivo identificar cuando y
    justificar la respuesta
  \end{cuestion}

  \begin{solucion}
     	cosas
  \end{solucion}
  \begin{cuestion}
    ¿Qué información de la imagen se codifica en el descriptor de SIFT?
    Justificar la contestación.
  \end{cuestion}

  \begin{solucion}
     	En el descriptor de SIFT se almacena un histograma de la orientación de los
      gradientes de cada píxel de una zona de la imagen con información característica.
  \end{solucion}

  \begin{cuestion}
    Describa un par de criterios que sirvan para establecer correspondencias
    (matching) entre descriptores de regiones extraídos de dos imágenes. Justificar
    la idoneidad de los mismos
  \end{cuestion}

  \begin{solucion}
     	cosas
  \end{solucion}

  \begin{cuestion}
    Cual es el objetivo principal en el uso de la técnica RANSAC. Justificar
    la respuesta
  \end{cuestion}

  \begin{solucion}
     	El objetivo principal es dar una alternativa de función de coste a la de mínimos
      cuadrados. Esto quiere decir que, matemáticamente, busca realizar una regresión
      lineal entre un conjunto de puntos en el que la minimización por mínimos cuadrados
      no es suficiente para conseguir una buena homografía (que es lo que buscamos con
      dicha regresión). Esto lo consigue trazando líneas aleatorias entre las muestras
      y tomar la que mejor nos convenga según un criterio de satisfacibilidad de las muestras.
  \end{solucion}
  \begin{cuestion}
    ¿Si tengo 4 imágenes de una escena de manera que se solapan la 1-2, 2-3
    y 3-4. ¿Cuál es el número mínimo de puntos en correspondencias necesarios para
    montar un mosaico? Justificar la respuesta
  \end{cuestion}

  \begin{solucion}
     	Para cada par de imágenes necesitamos como mínimo 4 imágenes. Este mínimo lo
      cumple el criterio RANSAC. Así que para 4 imágenes y 3 correspondencias, con
      4 puntos por correspondencia, necesitaremos 12 puntos en total como mínimo
      para montar un mosaico con dichas imágenes.
  \end{solucion}

  \begin{cuestion}
    En la confección de un mosaico con proyección rectangular es esperable
   que aparezcan deformaciones de la realidad. ¿Cuáles y porqué?.¿Bajo qué
   condiciones esas deformaciones podrían desaparecer? Justificar la respuesta

  \end{cuestion}

  \begin{solucion}
    Se esperan deformaciones de las imágenes más a los extremos. Esto es debido al
    \textit{error} que se va acumulando al realizar las proyecciones por homografías
    de coordenadas esféricas al plano. Para evitar estas deformaciones podemos usar
    una técnica común: empezar las proyecciones desde la imagen central y llegar
    hasta ambos extremos. Así las homografías no son tan fuertes y deforman menos la imagen.
    Pero en el contexto de que haya muchas imágenes con un cambio fuerte de coordenadas
    esféricas, la deformación será más clara y cada vez será más difícil realizar
    un mosaico con poca deformación.
  \end{solucion}


\end{document}
